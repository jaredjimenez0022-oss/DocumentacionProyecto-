%%%%%%%%%%%%%%%%%%%%%%%%%%%%%%%%%%%%%%%%%%%%%%%%%%
%%%%		~~~~ PACKAGE LISTING ~~~~
%%%%%%%%%%%%%%%%%%%%%%%%%%%%%%%%%%%%%%%%%%%%%%%%%%


% USED TO SET TEMPORARY TEXT
\usepackage{lipsum}

% neater appendix
\usepackage{appendix}

% Nicer ragged edges
\usepackage[document]{ragged2e}

% better representation of figures
\usepackage{graphicx}

% language packages & linebreaking
\usepackage[english]{babel}

% Computer science style quotes
\usepackage{csquotes}

% Makes Natbib and hyperref work together
\usepackage{hyphenat}

% slightly different styling of abstract
\usepackage{bookabstract}

% ADJUSTABLE BOXES
\usepackage{adjustbox}

% FOR USING LANDSCAPE PAGES
\usepackage{lscape}


% Filling out the text
\usepackage[parfill]{parskip}
\setlength\parskip{.5\baselineskip plus .1\baselineskip minus .1\baselineskip}



%%%%%%%%%%%%%%%%%%%%%%%%%%%%%%%%%%%%%%%%%%%%%%%%%%
%%%%		~~~~ GEOMETRY STYLE ~~~~
%%%%%%%%%%%%%%%%%%%%%%%%%%%%%%%%%%%%%%%%%%%%%%%%%%

% SET PAGE MARGINS
\usepackage[top=30mm, bottom=30mm, inner=35mm, outer=35mm, headsep=10mm, footskip=12mm]{geometry}

% INCREASES THE SPACE BETWEEN LINES, EASIER READABILITY
\usepackage{setspace}
\setstretch{1.4}


% SET MAXIMUM DEPTH OF TOC AND SECTIONS
\setcounter{secnumdepth}{4}
\setcounter{tocdepth}{1}

% SET SPACE BETWEEN TEXT AND FOOTNOTES
\setlength{\skip\footins}{8mm}

% SET PARAGRAPH INDENT
\setlength{\parindent}{0pt}
\setlength{\parskip}{2mm}

% This removes the forced empty pages
\let\cleardoublepage\clearpage


%%%%%%%%%%%%%%%%%%%%%%%%%%%%%%%%%%%%%%%%%%%%%%%%%%
%%%%		~~~~ HEADER STYLING ~~~~
%%%%%%%%%%%%%%%%%%%%%%%%%%%%%%%%%%%%%%%%%%%%%%%%%%

\addtolength{\headheight}{3pt}
\usepackage{fancyhdr}
\pagestyle{fancy}% <- must be used before the redefinition of \chaptermark and \sectionmark
% change the marks set by \chapter and \section
\renewcommand{\chaptermark}[1]{\markboth{#1}{}}
\renewcommand{\sectionmark}[1]{\markright{\thesection\ #1}}

% change fancy style
\fancypagestyle{fancy}{%
\fancyhf{}
\fancyhead[LE]{\nouppercase{\leftmark}}
\fancyhead[RO]{\nouppercase{\rightmark}}
\fancyfoot[RE,RO]{\thepage}
\renewcommand{\headrulewidth}{0.4pt}
}

% change plain style
\fancypagestyle{plain}{%
\fancyhf{}
% \fancyhead[LE]{\nouppercase{\leftmark}}
% \fancyhead[RO]{\nouppercase{\rightmark}}
\fancyfoot[RE,RO]{\thepage}
\renewcommand{\headrulewidth}{0pt}
}

% PAGESTYLE IN TOC
\fancypagestyle{toc}{%
  \fancyhf{}
  \fancyfoot[RE,RO]{\thepage}
}

%%%%%%%%%%%%%%%%%%%%%%%%%%%%%%%%%%%%%%%%%%%%%%%%%%
%%%%		~~~~ BIBLIOGRAPHY STYLE CHOICES ~~~~
%%%%%%%%%%%%%%%%%%%%%%%%%%%%%%%%%%%%%%%%%%%%%%%%%%

% BIBLIOGRAPHY STYLE 
\usepackage[
    backend=biber,
    style=ieee,
    sorting=none,
  ]{biblatex}
\addbibresource{bibliography.bib}

% MANAGE LONG BIBLIOGRAPHY URL
\setcounter{biburlnumpenalty}{9000}
\setcounter{biburlucpenalty}{9000}
\setcounter{biburllcpenalty}{9000}


% SET CHAPTER APPEARENCE (ADD NUMBER NEXT TO THE CHAPTER, MARGIN...)
\usepackage{titlesec}
\titleformat{\chapter}[hang]{\LARGE\bfseries}{\thechapter{. }}{0pt}{\LARGE\bfseries}
\titlespacing*{\chapter}{0pt}{-5pt}{40pt} 

%%%%%%%%%%%%%%%%%%%%%%%%%%%%%%%%%%%%%%%%%%%%%%%%%%
%%%%		~~~~ FIGURE CAPTION STYLE ~~~~
%%%%%%%%%%%%%%%%%%%%%%%%%%%%%%%%%%%%%%%%%%%%%%%%%%
% Enables controlling the look and feel of captions, see package documentation
\usepackage[font=small, labelfont=bf, margin=0.3cm]{caption}        

% Recommended when making sub-figures
\usepackage{subcaption}     

% Easily insert sources in images
\newcommand{\source}[1]{\vspace{-4pt} \caption*{\hfill \footnotesize{Source: {#1}} } }   

%%%%%%%%%%%%%%%%%%%%%%%%%%%%%%%%%%%%%%%%%%%%%%%%%%
%%%%		~~~~ HYPERLINK STYLE ~~~~
%%%%%%%%%%%%%%%%%%%%%%%%%%%%%%%%%%%%%%%%%%%%%%%%%%

% Change hyperref colors
\usepackage[pdfpagelabels=true]{hyperref}
\hypersetup{
pdftitle={Thesis title},
pdfsubject={MasteR thesis},
pdfauthor={Author name},
pdfkeywords={keyword1, keyword2, keyword3}
}

% Change hyperlink colors
\usepackage{xcolor}
\definecolor{blulink}{HTML}{0F4ACD}
\definecolor{redcite}{HTML}{E34D51}
\hypersetup{
    colorlinks,
    linkcolor={black}, % CHANGE COLOR IF NEEDED
    citecolor={black}, % CHANGE COLOR IF NEEDED
    urlcolor={black}  % CHANGE COLOR IF NEEDED
}

%%%%%%%%%%%%%%%%%%%%%%%%%%%%%%%%%%%%%%%%%%%%%%%%%%
%%%%		~~~~ MATHEMATICAL STYLE ~~~~
%%%%%%%%%%%%%%%%%%%%%%%%%%%%%%%%%%%%%%%%%%%%%%%%%%

% Makes math appear bold
\usepackage{bm}         

% Add math options and tools
\usepackage{amsmath}
\usepackage{mathtools}

% Extended symbol collection
\usepackage{amssymb}    

% Helps define theorem-like structures
\usepackage{amsthm}     

% Used in the package "gensymb" (below), which will give warnings if "textcomp" is not imported in advance
\usepackage{textcomp}   

% Adds extra generic symbols for math and text mode, e.g. \degree
\usepackage{gensymb}    



% Set space above and below a math enviroment
\makeatletter
\g@addto@macro\normalsize{%
  \setlength\abovedisplayskip{8mm}
  \setlength\belowdisplayskip{8mm}
  \setlength\abovedisplayshortskip{6mm}
  \setlength\belowdisplayshortskip{6mm}
}
\makeatother

% Nicer math usage
\usepackage{siunitx}
\sisetup{exponent-product=\ensuremath{{}\cdot{}}}
