%%%%%%%%%%%%%%%%%%%%%%%%%%%%%%%%%%%%%%%%%%%%%%%%%%
%%%%		~~~~ Results ~~~~
%%%%%%%%%%%%%%%%%%%%%%%%%%%%%%%%%%%%%%%%%%%%%%%%%%

\chapter{Análisis de Resultados}
\label{chap:results}
\pagestyle{fancy}

\section{Resultados por Modelo}

\subsection{Detección de Emociones}

\begin{table}[h]
\centering
\begin{tabular}{|l|c|c|c|}
\hline
\textbf{Emoción} & \textbf{Precision} & \textbf{Recall} & \textbf{F1-Score} \\
\hline
Felicidad & 0.89 & 0.91 & 0.90 \\
Tristeza & 0.82 & 0.79 & 0.80 \\
Enojo & 0.84 & 0.83 & 0.83 \\
Sorpresa & 0.87 & 0.88 & 0.87 \\
Miedo & 0.78 & 0.76 & 0.77 \\
Disgusto & 0.81 & 0.80 & 0.80 \\
Neutral & 0.86 & 0.87 & 0.86 \\
\hline
\textbf{Promedio} & \textbf{0.84} & \textbf{0.83} & \textbf{0.83} \\
\hline
\end{tabular}
\caption{Métricas de clasificación para detección de emociones}
\label{tab:emotion_metrics}
\end{table}

\textbf{Análisis}: El modelo alcanza 83\% de F1-score promedio, con mejor desempeño en emociones de alta valencia (felicidad, sorpresa). Las emociones negativas de baja intensidad (miedo) presentan mayor confusión, consistente con literatura \cite{mollahosseini2017affectnet}.

\subsection{Modelos de Regresión}

\begin{table}[h]
\centering
\begin{tabular}{|l|c|c|c|}
\hline
\textbf{Modelo} & \textbf{MAE} & \textbf{RMSE} & \textbf{R²} \\
\hline
Bitcoin & \$2,450 & \$3,200 & 0.78 \\
S\&P 500 & \$8.5 & \$12.3 & 0.76 \\
Aguacate & \$0.18 & \$0.24 & 0.71 \\
Autos & \$1,200 & \$1,850 & 0.82 \\
IMC/Grasa & 2.3\% & 3.1\% & 0.88 \\
Chicago Crime & 45 casos & 68 casos & 0.72 \\
Londres Crime & 12 casos & 18 casos & 0.68 \\
\hline
\end{tabular}
\caption{Métricas de regresión para modelos predictivos}
\label{tab:regression_metrics}
\end{table}

\textbf{Análisis}: 
\begin{itemize}
    \item \textbf{Mejor desempeño}: Predicción de IMC (R²=0.88) debido a relación directa altura-peso-grasa
    \item \textbf{Mayor volatilidad}: Bitcoin (MAE=\$2,450) refleja naturaleza estocástica de criptomonedas
    \item \textbf{Crímenes urbanos}: R² moderado (0.68-0.72) indica factores socioeconómicos no capturados
\end{itemize}

\subsection{Modelos de Clasificación}

\begin{table}[h]
\centering
\begin{tabular}{|l|c|c|c|}
\hline
\textbf{Modelo} & \textbf{Accuracy} & \textbf{Precision} & \textbf{Recall} \\
\hline
Cirrosis Hepática & 0.83 & 0.82 & 0.81 \\
Retrasos de Vuelos & 0.79 & 0.77 & 0.78 \\
\hline
\end{tabular}
\caption{Métricas de clasificación}
\label{tab:classification_metrics}
\end{table}

\section{Análisis de Rendimiento del Sistema}

\subsection{Latencia End-to-End}

\begin{itemize}
    \item \textbf{Detección emocional}: 280ms promedio (incluye captura + inferencia + render)
    \item \textbf{Speech-to-Text}: 450ms para comandos de 3-5 segundos
    \item \textbf{Predicciones ML}: 120-350ms según complejidad del modelo
    \item \textbf{Latencia total}: < 500ms en 95\% de requests
\end{itemize}

\subsection{Uso de Recursos}

\begin{itemize}
    \item \textbf{Memoria Backend}: 850MB con todos los modelos cargados
    \item \textbf{CPU}: 35\% utilización promedio (4 cores)
    \item \textbf{Tamaño modelos}: 245MB total (11 modelos serializados)
\end{itemize}

\section{Limitaciones Identificadas}

\begin{enumerate}
    \item \textbf{Detección emocional}: Baja precisión en condiciones de iluminación extrema
    \item \textbf{Series temporales}: Horizon de predicción limitado a 365 días (degradación R² > 1 año)
    \item \textbf{Speech-to-Text}: Requiere conexión a internet para Azure API (sin modo offline)
    \item \textbf{Escalabilidad}: Carga secuencial de modelos limita concurrencia (max 50 req/seg)
\end{enumerate}

\section{Comparación con Estado del Arte}

\begin{itemize}
    \item DeepFace (83\% F1) vs. SOTA FER+ (87\% F1) - Gap de 4pp debido a modelo base
    \item Random Forest vs. LSTM para series temporales: RF más eficiente (3x más rápido) con R² comparable
    \item Arquitectura monolítica vs. microservicios: 40\% reducción en latencia mediante paralelización
\end{itemize}

%\begin{figure}[h]
   % \centering
    %\includegraphics[width=0.8\textwidth]{IMAGES/confusion_matrix_emotions.png}
    %\caption{Matriz de confusión para clasificación de emociones}
   % \label{fig:confusion_matrix}
%\end{figure}