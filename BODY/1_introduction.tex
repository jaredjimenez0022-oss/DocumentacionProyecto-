%%%%%%%%%%%%%%%%%%%%%%%%%%%%%%%%%%%%%%%%%%%%%%%%%%
%%%%		~~~~ Introduction ~~~~
%%%%%%%%%%%%%%%%%%%%%%%%%%%%%%%%%%%%%%%%%%%%%%%%%%

\chapter{Introducción}
\label{chap:intro}
\pagestyle{fancy}

\section{Motivación y Contexto}

En la era de la transformación digital, los asistentes virtuales han pasado de ser simples ejecutores de comandos a sistemas inteligentes capaces de comprender contexto emocional y proporcionar análisis predictivos complejos. En el ámbito educativo del Instituto Tecnológico de Costa Rica, existe la oportunidad de desarrollar herramientas que faciliten la interacción estudiante-tecnología mediante interfaces naturales y multimodales.

El proyecto \textbf{Jarvis TEC} surge como respuesta a la necesidad de un asistente inteligente que integre:
\begin{itemize}
    \item Reconocimiento y análisis de emociones faciales en tiempo real
    \item Procesamiento de comandos de voz con alta precisión
    \item Capacidades predictivas en múltiples dominios (financiero, médico, social)
    \item Arquitectura escalable y modular basada en microservicios
\end{itemize}

\section{Revisión de Literatura}

\subsection{Reconocimiento de Emociones}

El reconocimiento automático de emociones faciales ha sido ampliamente estudiado utilizando Redes Neuronales Convolucionales (CNN). DeepFace representa el estado del arte en detección facial y análisis emocional, utilizando arquitecturas VGG-Face y ResNet para extraer características faciales profundas. Estudios recientes  demuestran que sistemas ensemble que combinan múltiples detectores faciales alcanzan precisiones superiores al 80\% en datasets como FER+ y AffectNet.

\subsection{Speech-to-Text y Procesamiento de Voz}

Azure Cognitive Services \cite{azure2023speech} proporciona modelos de Speech-to-Text basados en arquitecturas Transformer que logran tasas de error de palabras (WER) inferiores al 5\% en idiomas como español e inglés. La integración de modelos como Whisper \cite{radford2022whisper} ha democratizado el acceso a reconocimiento de voz de alta calidad.

\subsection{Ensemble Learning para Predicción}

Random Forest y Gradient Boosting han demostrado ser altamente efectivos en tareas de regresión y clasificación. Su robustez ante overfitting y capacidad de manejar features no lineales los hace ideales para predicciones financieras, médicas y sociales.

\section{Formulación del Problema}

¿Cómo diseñar e implementar un asistente inteligente multimodal que integre reconocimiento facial, procesamiento de voz y múltiples modelos predictivos especializados, manteniendo baja latencia y alta precisión en tiempo real?

\subsection{Objetivos Específicos}

\begin{enumerate}
    \item Implementar un sistema de detección de emociones faciales con precisión superior al 80\%
    \item Desarrollar pipeline de Speech-to-Text con soporte para comandos en español
    \item Entrenar y desplegar al menos 10 modelos de Machine Learning especializados
    \item Diseñar arquitectura de microservicios con latencia inferior a 500ms
    \item Crear interfaz web intuitiva para interacción multimodal
\end{enumerate}