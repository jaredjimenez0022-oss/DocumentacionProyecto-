%%%%%%%%%%%%%%%%%%%%%%%%%%%%%%%%%%%%%%%%%%%%%%%%%%
%%%%		~~~~ Data ~~~~
%%%%%%%%%%%%%%%%%%%%%%%%%%%%%%%%%%%%%%%%%%%%%%%%%%

\chapter{Datos}
\label{chap:data}
\pagestyle{fancy}

\section{Descripción de los Conjuntos de Datos}

El proyecto utiliza 11 conjuntos de datos públicos y privados para entrenar modelos especializados:

\subsection{Datos para Detección de Emociones}

\textbf{DeepFace}: Utiliza modelos pre-entrenados en FER2013 y VGGFace2, datasets que contienen más de 35,000 imágenes faciales etiquetadas con 7 emociones básicas (felicidad, tristeza, enojo, sorpresa, miedo, disgusto, neutral).

\subsection{Datos Financieros}

\begin{itemize}
    \item \textbf{Bitcoin Historical Data}: Serie temporal de 2010-2024 con precios diarios, volumen y capitalización de mercado (9,500+ registros)
    \item \textbf{S\&P 500 Dataset}: Datos de 505 empresas durante 5 años con métricas OHLCV (Open, High, Low, Close, Volume)
    \item \textbf{Avocado Prices}: 18,000 registros de precios de aguacate por región y tipo en Estados Unidos
\end{itemize}

\subsection{Datos Médicos}

\begin{itemize}
    \item \textbf{Cirrhosis Dataset}: 424 pacientes con 19 características clínicas (bilirrubina, albumina, edad, etc.)
    \item \textbf{Body Fat Dataset}: 252 registros con mediciones antropométricas para predicción de IMC
\end{itemize}

\subsection{Datos Sociales}

\begin{itemize}
    \item \textbf{Chicago Crime Dataset}: BigQuery público con millones de incidentes criminales (2001-2024)
    \item \textbf{London Crime Dataset}: Datos geoespaciales de crímenes por LSOA (Lower Layer Super Output Area)
\end{itemize}

\subsection{Datos de Transporte y Comercio}

\begin{itemize}
    \item \textbf{Airline Delay Dataset}: 5.8M de vuelos con características de retraso
    \item \textbf{Car Price Dataset}: 301 registros de autos usados con 9 características (año, kilometraje, tipo de combustible)
    \item \textbf{MovieLens Dataset}: 100,000 ratings de 9,000 películas para sistema de recomendación
\end{itemize}

\section{Preparación y Procesamiento}

\subsection{Limpieza de Datos}

Para cada dataset se aplicaron las siguientes técnicas:
\begin{itemize}
    \item Eliminación de valores nulos mediante imputación por mediana (variables numéricas) o moda (categóricas)
    \item Detección y remoción de outliers usando IQR (Interquartile Range)
    \item Normalización de series temporales con diferenciación para lograr estacionariedad
\end{itemize}

\subsection{Ingeniería de Características}

\textbf{Series Temporales}: Creación de lag features (t-1, t-3, t-7) y rolling means (7, 30 días) para capturar patrones temporales.

\textbf{Variables Categóricas}: Label Encoding para variables ordinales; One-Hot Encoding para nominales con baja cardinalidad.

\textbf{Feature Scaling}: Aplicación de StandardScaler para modelos sensibles a escala.

\subsection{Consideraciones Éticas}

\begin{itemize}
    \item Todos los datasets utilizados son de dominio público o con licencias permisivas
    \item Los datos médicos están anonimizados según estándares HIPAA
    \item El sistema de detección emocional NO almacena imágenes de usuarios, solo metadatos
    \item Cumplimiento con GDPR para datos de usuarios europeos (datasets de Londres)
\end{itemize}

%\begin{figure}[h]
   % \centering
   % \includegraphics[width=0.8\textwidth]{IMAGES/data_distribution.png}
   % \caption{Distribución de registros por dataset utilizado}
   % \label{fig:data_distribution}
%\end{figure}