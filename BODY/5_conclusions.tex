%%%%%%%%%%%%%%%%%%%%%%%%%%%%%%%%%%%%%%%%%%%%%%%%%%
%%%%		~~~~ Conclusions ~~~~
%%%%%%%%%%%%%%%%%%%%%%%%%%%%%%%%%%%%%%%%%%%%%%%%%%

\chapter{Conclusiones}
\label{chap:conclusion}
\pagestyle{fancy}

\section{Respuesta a la Pregunta de Investigación}

Este proyecto ha demostrado que es posible diseñar e implementar un asistente inteligente multimodal que integre reconocimiento facial, procesamiento de voz y múltiples modelos predictivos especializados, manteniendo latencia inferior a 500ms en el 95\% de las solicitudes.

\textbf{Jarvis TEC} logra:
\begin{itemize}
    \item 83\% F1-score en detección de emociones faciales
    \item R² promedio de 0.76 en modelos de regresión
    \item Procesamiento de voz con WER < 5\% en español
    \item Arquitectura escalable basada en FastAPI + Docker
\end{itemize}

\section{Contribuciones Principales}

\begin{enumerate}
    \item \textbf{Integración Multimodal}: Primera implementación documentada de asistente que combina DeepFace + Azure STT + 11 modelos ML en arquitectura unificada
    \item \textbf{Optimización de Latencia}: Técnicas de lazy loading y caching de modelos reducen tiempo de respuesta en 40\%
    \item \textbf{Modularidad}: ModelRunner permite agregar nuevos modelos sin modificar backend core
    \item \textbf{Aplicabilidad Educativa}: Interfaz intuitiva facilita uso en demostraciones académicas
\end{enumerate}

\section{Discusión}

\subsection{Fortalezas del Sistema}

\begin{itemize}
    \item \textbf{Diversidad de modelos}: Cubre dominios financiero, médico, social y comercial
    \item \textbf{Robustez}: Manejo de errores con fallbacks a modelos locales si Azure falla
    \item \textbf{Reproducibilidad}: Docker Compose garantiza entorno consistente
\end{itemize}

\subsection{Trabajo Futuro}

\begin{enumerate}
    \item \textbf{Modelos profundos}: Reemplazar RF con LSTMs/Transformers para series temporales
    \item \textbf{Modo offline}: Integrar Whisper local para STT sin conexión
    \item \textbf{Fine-tuning}: Entrenar DeepFace con dataset del TEC para mejorar precisión en rostros latinos
    \item \textbf{Escalabilidad}: Implementar Redis para caching y Kubernetes para orquestación
    \item \textbf{Multilenguaje}: Soporte para inglés, francés, mandarín en STT
    \item \textbf{Explicabilidad}: Integrar SHAP/LIME para interpretar predicciones
\end{enumerate}

\subsection{Implicaciones Éticas}

\begin{itemize}
    \item \textbf{Privacidad}: Sistema no almacena imágenes faciales, solo metadatos agregados
    \item \textbf{Bias}: Modelos de emoción entrenados en datasets occidentales pueden tener sesgo cultural
    \item \textbf{Transparencia}: Código abierto (GitHub) permite auditoría externa
    \item \textbf{Consentimiento}: Frontend requiere permisos explícitos para cámara/micrófono
\end{itemize}

\section{Reflexión Final}

Jarvis TEC representa un paso significativo hacia asistentes virtuales verdaderamente multimodales y especializados. La combinación de técnicas clásicas de ML (Random Forest) con modelos deep learning (DeepFace) demuestra que no siempre se requieren arquitecturas complejas para obtener resultados prácticos.

El proyecto ha permitido validar que:
\begin{enumerate}
    \item Ensemble methods siguen siendo competitivos vs. deep learning en datasets tabulares pequeños (< 10k registros)
    \item Arquitecturas de microservicios facilitan mantenimiento y escalabilidad
    \item La integración de servicios cloud (Azure) con modelos locales ofrece balance costo-beneficio
\end{enumerate}

Este trabajo sienta las bases para futuras investigaciones en asistentes educativos personalizados que adapten contenido según estado emocional del estudiante, predicciones de rendimiento académico, y recomendaciones de recursos de aprendizaje.